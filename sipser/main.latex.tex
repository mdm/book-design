\documentclass{book}

\usepackage{layouts}

\usepackage{geometry}
\geometry{papersize={165mm,238mm}}

%%% fonts
\usepackage{fontspec}

\setmainfont{Kis Classico}
\setsansfont{Copperplate Gothic}
% \setmonofont{}

% \newfontfamily{\pagenumberfamily}{Noyh Black}
\newfontfamily{\pagenumberfamily}{Copperplate Gothic}


\usepackage{graphicx}
\usepackage{lipsum}

% \AddToHook{shipout/background}{%
%     \put (0in,-\paperheight){%
%         \includegraphics[width=\paperwidth,height=\paperheight,page=7]{%
%             reference.pdf%
%         }%
%     }%
% }

\setcounter{secnumdepth}{1}
\setcounter{chapter}{-1}

\begin{document}

\tableofcontents

\chapter{Preface to the First Edition}
\subsection{To the student}
\subsection{To the educator}
\subsection{The first edition}
\subsection{Feedback to the author}
\subsection{Acknowledgments}

\chapter{Preface to the Second Edition}
\chapter{Preface to the Third Edition}

\chapter{Introduction}
\section{Automata, Computability, and Complexity}
\subsection{Complexity theory}
\subsection{Computability theory}
\subsection{Automata theory}
\section{Mathematical Notions and Terminology}
\subsection{Sets}
\subsection{Sequences and tuples}
\subsection{Functions and relations}
\subsection{Graphs}
\subsection{Strings and languages}
\subsection{Boolean logic}
\subsection{Summary of mathematical terms}
\section{Definitions, Theorems, and Proofs}
\subsection{Finding proofs}
\section{Types of Proof}
\subsection{Proof by construction}
\subsection{Proof by contradiction}
\subsection{Proof by induction}
\subsection{Exercises, Problems, and Solutions}


\part{Automata and Languages}
\chapter{Regular Languages}
\section{Finite Automata}
\subsection{Formal definition of a finite automaton}
\subsection{Examples of finite automata}
\subsection{Formal definition of computation}
\subsection{Designing finite automata}
\subsection{The regular operations}
\section{Nondeterminism}
\subsection{Formal definition of a nondeterministic finite automaton}
\subsection{Equivalence of NFAs and DFAs}
\subsection{Closure under the regular operations}
\section{Regular Expressions}
\subsection{Formal definition of a regular expression}
\subsection{Equivalence with finite automata}
\section{Nonregular Languages}
\subsection{The pumping lemma for regular languages}
\subsection{Exercises, Problems, and Solutions}

\chapter{Context-Free Languages}
\section{Context-Free Grammars}
\subsection{Formal definition of a context-free grammar}
\subsection{Examples of context-free grammars}
\subsection{Designing context-free grammars}
\subsection{Ambiguity}
\subsection{Chomsky normal form}
\section{Pushdown Automata}
\subsection{Formal definition of a pushdown automaton}
\subsection{Examples of pushdown automata}
\subsection{Equivalence with context-free grammars}
\section{Non-Context-Free Languages}
\subsection{The pumping lemma for context-free languages}
\section{Deterministic Context-Free Languages}
\subsection{Properties of DCFLs}
\subsection{Deterministic context-free grammars}
\subsection{Relationship of DPDAs and DCFGs}
\subsection{Parsing and LR(k) Grammars}
\subsection{Exercises, Problems, and Solutions}

\part{Computability Theory}
\chapter{The Church–Turing Thesis}
\section{Turing Machines}
\subsection{Formal definition of a Turing machine}
\subsection{Examples of Turing machines}
\section{Variants of Turing Machines}
\subsection{Multitape Turing machines}
\subsection{Nondeterministic Turing machines}
\subsection{Enumerators}
\subsection{Equivalence with other models}
\section{The Definition of Algorithm}
\subsection{Hilbert’s problems}
\subsection{Terminology for describing Turing machines}
\subsection{Exercises, Problems, and Solutions}

\chapter{Decidability}
\section{Decidable Languages}
\subsection{Decidable problems concerning regular languages}
\subsection{Decidable problems concerning context-free languages}
\section{Undecidability}
\subsection{The diagonalization method}
\subsection{An undecidable language}
\subsection{A Turing-unrecognizable language}
\subsection{Exercises, Problems, and Solutions}

\chapter{Reducibility}
\section{Undecidable Problems from Language Theory}
\subsection{Reductions via computation histories}
\section{A Simple Undecidable Problem}
\section{Mapping Reducibility}
\subsection{Computable functions}
\subsection{Formal definition of mapping reducibility}
\subsection{Exercises, Problems, and Solutions}

\chapter{Advanced Topics in Computability Theory}
\section{The Recursion Theorem}
\subsection{Self-reference}
\subsection{Terminology for the recursion theorem}
\subsection{Applications}
\section{Decidability of logical theories}
\subsection{A decidable theory}
\subsection{An undecidable theory}
\section{Turing Reducibility}
\section{A Definition of Information}
\subsection{Minimal length descriptions}
\subsection{Optimality of the definition}
\subsection{Incompressible strings and randomness}
\subsection{Exercises, Problems, and Solutions}

\part{Complexity Theory}
\chapter{Time Complexity}
\section{Measuring Complexity}
\subsection{Big-O and small-o notation}
\subsection{Analyzing algorithms}
\subsection{Complexity relationships among models}
\section{The Class P}
\subsection{Polynomial time}
\subsection{Examples of problems in P}
\section{The Class NP}
\subsection{Examples of problems in NP}
\subsection{The P versus NP question}
\section{NP-completeness}
\subsection{Polynomial time reducibility}
\subsection{Definition of NP-completeness}
\subsection{The Cook–Levin Theorem}
\section{Additional NP-complete Problems}
\subsection{The vertex cover problem}
\subsection{The Hamiltonian path problem}
\subsection{The subset sum problem}
\subsection{Exercises, Problems, and Solutions}

\chapter{Space Complexity}
\section{Savitch’s Theorem}
\section{The Class PSPACE}
\section{PSPACE-completeness}
\subsection{The TQBF problem}
\subsection{Winning strategies for games}
\subsection{Generalized geography}
\section{The Classes L and NL}
\section{NL-completeness}
\subsection{Searching in graphs}
\section{NL equals coNL}
\subsection{Exercises, Problems, and Solutions}

\chapter{Intractability}
\section{Hierarchy Theorems}
\subsection{Exponential space completeness}
\section{Relativization}
\subsection{Limits of the diagonalization method}
\section{Circuit Complexity}
\subsection{Exercises, Problems, and Solutions}

\chapter{Advanced Topics in Complexity Theory}
\section{Approximation Algorithms}
\section{Probabilistic Algorithms}
\subsection{The class BPP}
\subsection{Primality}
\subsection{Read-once branching programs}
\section{Alternation}
\subsection{Alternating time and space}
\subsection{The Polynomial time hierarchy}
\section{Interactive Proof Systems}
\subsection{Graph nonisomorphism}
\subsection{Definition of the model}
\subsection{IP = PSPACE}
\section{Parallel Computation}
\subsection{Uniform Boolean circuits}
\subsection{The class NC}
\subsection{P-completeness}
\section{Cryptography}
\subsection{Secret keys}
\subsection{Public-key cryptosystems}
\subsection{One-way functions}
\subsection{Trapdoor functions}
\subsection{Exercises, Problems, and Solutions}

\chapter{Selected Bibliography}
\chapter{Index}

\end{document}
